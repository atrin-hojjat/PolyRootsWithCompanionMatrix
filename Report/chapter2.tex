\chapter{روش ماتریس همراه}

در این فصل، روش ماتریس همراه را معرفی می‌کنیم.
این روش، قابل اعتمار ترین روش برای محاسبه‌ی ریشه‌های تکراری یک چند جمله‌ای ست.
هردو روش بررسی شده در بالا، با فرض عدم وجود ریشه‌های تکراری کار می‌کنند.
محاسبه‌ی این ریشه‌ها در عملیات‌های کامپیوتر معمولا خطای زیادی دارد.
از این رو پیدا کردن روشی که بتواند این ریشه‌ها را نیز با دقت خوبی محاسبه کند بسیار حائز اهمیت است.

\section{بررسی روش ماتریس همراه}
فرض کنید چند جمله‌ای
$p$
به فرم

\begin{equation}
  p(x) = x^n + a_{n-1} x^{n-1} + \cdots + a_1 x + a_0
\end{equation}

داده شده است.
ماتریس همراه آن را به شکل زیر تعریف می‌کنیم:

\begin{equation}
  C =
    \begin{bmatrix}
      0 & 0 & 0 & \cdots & 0 & -a_0 \\
      1 & 0 & 0 & \cdots & 0 & -a_1 \\
      0 & 1 & 0 & \cdots & 0 & -a_2 \\
      0 & 0 & 1 & \cdots & 0 & -a_3 \\
      \vdots &  &  & \ddots & & \vdots \\
      0 & 0 & 0 & \cdots & 1 & -a_{n-1} \\

    \end{bmatrix}
\end{equation}


در این روش، مقادری ویژه‌ی ماتریس
$C$
همان ریشه‌های چندجمله‌ای
$p$
خواهد بود.

این روش سریع‌ترین یا بهینه ترین روش برای محاسبه‌ی ریشه‌های چند جمله‌ای نیست، از آنجایی که
نیاز به
$O(n^2)$
حافظه و
$O(n^3)$
محاسبه دارد، در حالی که یک الگوریتم بهینه برای به دست آوردن ریشه‌های چندجمله‌ای میتواند با حافظه
$O(n)$
و پیچیدگی محاسباتی
$O(n^2)$
کار کند.
از طرفی، هرچه تعداد محاسبات بیشتر باشد، میزان خطای این روش نیز بیشتر خواهد بود~\cite{moler1991cleve}
.

در محاسبه‌ی مقادیر ویژه‌ی یک ماتریس، میزان خطا در کامپیوتر حداقل از اردر
$O(\epsilon ||A||_F)$
که
$\epsilon$
دقت ماشین و
$||A||_F$
نرم فروبنیوس ماتریس است~\cite{osborne1960pre}
.
برای کاهش این خطا، می‌توان از تغییرات قطری
\footnote{\lr{Diagonal similarity transformations}}
استفاده کرد بطوری‌که نرم
$A$
کاهش یابد~\cite{beresford1969}~\cite{james2014matrix}~\cite{osborne1960pre}
.

پس از نرمال کردن ماتریس، می‌توان از الگوریتم
\lr{QR}
برای به دست آوردن مقادیر ویژه‌ی ماتریس استفاده کرد.
در این الگوریتم، در مرحله
$k$
ام،
ابتدا تجزیه‌ی
\lr{QR}
برای ماتریس
$A_k$
به فرم
$A_k = Q_k R_k$
را به
دست می‌آوریم که
$A_0 = A$
.
سپس در هر مرحله
قرار می‌دهیم
$A_{k + 1} = R_k Q_k$
.
با توجه به معدله‌ی زیر، می‌توان دید که در هر مرحله،
$A_k$
مشابه
$A$
است.

\begin{equation}
  A_{k+1} = R_k Q_k = Q_{k}^{-1} Q_k R_k Q_k = Q_k^{-1} A Q_k = Q_k^T A Q_k
\end{equation}

پس از تعدادی مرحله، ماتریس
$A_k$
به ماتریسی مثلثی تبدیل می‌شود که به آن شکل شور
\footnote{\lr{Schor form of $A$}}
$A$
گفته می‌شود.
مقادیر ویژه‌ی ماتریس‌های مثلثی، برابر مقادیر روی قطر آنهاست. با توجه به اینکه
$A$
و
$A_k$
مشابه‌اند، مقادیر ویژه‌ی
$A$
نیز به دست آمده است.
ممکن است شرایطی به وجود آید که
$A_k$
مثلثی نشود~\cite{golub2013matrix}
.


\section{اثبات روش ماتریس همراه}

به منظور اثبات درستی این روش، کافیست نشان دهیم که چندجمله‌ای مشخصه‌ی ماتریس
$C$
به فرم
$det(Ix - C)$
برابر چندجمله‌ای
$p(x)$
است.

برای اثبات صحت این روش، از استقرا استفاده می‌کنیم:

\begin{proof}
  پایه: $\left(n=1\right)$
  $$ A = [a_0], det(Ix - A) = x - a_0 $$

  گام:
  \begin{equation}
    Ix - C =
    \begin{bmatrix}
      x & 0 & 0 & \cdots & 0 & a_0 \\
      -1 & x & 0 & \cdots & 0 & a_1 \\
      0 & -1 & x & \cdots & 0 & a_2 \\
      0 & 0 & -1 & \cdots & 0 & a_3 \\
      \vdots &  &  & \ddots & & \vdots \\
      0 & 0 & 0 & \cdots & -1 & x + a_{n-1} \\
    \end{bmatrix}
  \end{equation}
  \begin{equation}
    \begin{split}
      det(Ix - C) =
      \begin{vmatrix}
        x & 0 & 0 & \cdots & 0 & a_0 \\
        -1 & x & 0 & \cdots & 0 & a_1 \\
        0 & -1 & x & \cdots & 0 & a_2 \\
        0 & 0 & -1 & \cdots & 0 & a_3 \\
        \vdots &  &  & \ddots & & \vdots \\
        0 & 0 & 0 & \cdots & -1 & x + a_{n-1} \\
      \end{vmatrix}
      \\
      =
      x
      \begin{vmatrix}
        x & 0 & \cdots & 0 & a_1 \\
        -1 & x & \cdots & 0 & a_2 \\
        0 & -1 & \cdots & 0 & a_3 \\
        \vdots &  &  & \ddots & & \vdots \\
        0 & 0 & \cdots & -1 & x + a_{n-1} \\
      \end{vmatrix}
      -
      (-1)^{n+1}a_0
      \begin{vmatrix}
        x & 0 & \cdots & 0 \\
        -1 & x & \cdots & 0 \\
        0 & -1 & \cdots & 0 \\
        \vdots &  & \ddots & \vdots \\
        0 & 0 & \cdots & -1 \\
      \end{vmatrix}
    \end{split}

  \end{equation}

  طبق فرض استقرا داریم:

  \begin{equation}
    \begin{vmatrix}
      x & 0 & \cdots & 0 & a_1 \\
      -1 & x & \cdots & 0 & a_2 \\
      0 & -1 & \cdots & 0 & a_3 \\
      \vdots &  &  & \ddots & \vdots \\
      0 & 0 & \cdots & -1 & x + a_{n-1} \\
    \end{vmatrix}
    = x^{n-1} + a_{n-1}x^{n-2} + \cdots + a_2 x + a_1
  \end{equation}
  و دترمینان ماتریس دوم به دلیل مثلثی بودن، برابر با حاصل ضرب مقادیر روی قطر آن است

  \begin{equation}
    \begin{split}
        det(Ix - C) = x (x^{n-1} + a_{n-1}x^{n-2} + \cdots + a_2 x + a_1) + (-1)^{n + 1} a_0 (-1)^{n-1} \\
        = x^n + a_{n-1} x^{n-1} + \cdots + a_1x + a_0
    \end{split}
  \end{equation}

\end{proof}


\section{محدود کردن بازه‌ی ریشه‌های معادله}

همانطور که در بخش ۱ دیدیم، روش‌هایی مانند روش نیوتون نیازمند وجود یک تخمین از محدوده‌ی ریشه‌هاست.
برای مثال، روش نیوتون نیاز داشت که حداکثر مقدار ریشه‌های چندجمله‌ای را داشته باشد، تا حدس اولیه‌اش بزرگ تر از بزرگترین ریشه‌ی چندجمله‌ای باشد.
در این بخش می‌خواهیم استفاده‌ی ماتریس همراه را در تخمین مقادیر ریشه‌ها بررسی کنیم.

در فصل ۵ درس، دیدیم که می‌توان از حلقه‌های گرشگروین
\footnote{\lr{Gershgorin}}
استفاده کرد تا محدوده‌ی مقادیر ویژه‌ی یک ماتریس را به دست آورد.
شعاع هر یک از این
$n$
حلقه از رابطه‌ی
\begin{equation}
  r_i = \sum\limits_{
    \substack{
      j = 1 \\
      i \ne j
    }
  }^{n} |a_{ij}|
\end{equation}
هر یک از مقادیر ویژه‌ی ماتریس
$A$
باید حداقل در یکی از نامساوی‌های زیر صدق کند:
\begin{equation}
  | \lambda - a_{ii} | \le r_i
\end{equation}

از آنجایی که

\begin{equation}
  C =
    \begin{bmatrix}
      0 & 0 & 0 & \cdots & 0 & -a_0 \\
      1 & 0 & 0 & \cdots & 0 & -a_1 \\
      0 & 1 & 0 & \cdots & 0 & -a_2 \\
      0 & 0 & 1 & \cdots & 0 & -a_3 \\
      \vdots &  &  & \ddots & & \vdots \\
      0 & 0 & 0 & \cdots & 1 & -a_{n-1} \\

    \end{bmatrix}
\end{equation}

داریم

\begin{equation}
  \begin{cases}
    | \lambda | \le a_i & \forsome 0 \le i < n - 1 \text{\lr{or}} \\
    | \lambda + a_{n-1} | \le 1 & \text{\lr{else}}.
  \end{cases}
\end{equation}

یعنی یک دیسک با شعاع
$1$
در مرکز
$a_{n-1}$
و
$n-2$
دیسک با مرکز
$0$
قرار می‌گیرند.
پس می‌توان گفت که ریشه‌های یک چندجمله‌ای دی یکی از دو بازه‌ی زیر اند
\begin{equation}
  \begin{cases}
    | \lambda | \le \max_{0 \le i < n - 1} a_i & \text{\lr{or}} \\
    | \lambda + a_{n-1} | \le 1 & \text{\lr{else}}.
  \end{cases}
\end{equation}

\insertfig{../output/gershgorin.pgf}{\lr{Gershgorin Discs for a Companion Matrix}}{GERSHGORIN_CMP_MAT}


\section{پیدا کردن کوچکترین یا بزرگترین ریشه‌ی یک چندجمله‌ای با کمک روش توانی}

می‌دانیم یک روش برای محاسبه‌ی یک مقدار ویژه‌ی ماتریس، استفاده از روش توانی است.
در این ورش، میتوان بصورت تکراری با شروع از یک بردار دلخواه، بزرگترین مقدار ویژه‌ی یک ماتریس را محاسبه‌کرد.

در این روش با شروع از
$V^{(0)}$
در هر مرحله

\begin{equation}
  V^{(i)} = A V^{(i - 1)}
\end{equation}

برای کاهش خطای محاسبه می‌توان قرار داد:
\begin{equation}
  \begin{split}
    V^{(i)} = A \tilde{V}^{(i - 1)} \\
    \tilde{V}^{(i)} = V^{(i)} / || V^{(i)} ||_\infty
  \end{split}
\end{equation}

بطوری که


\begin{equation}
  \begin{split}
    \lambda \approx \frac{\tilde{V}_j^{(i)}}{\tilde{V}_j^{(i - 1)}}
  \end{split}
\end{equation}

برای محاسبه‌ی کوچکترین مقدار ویژه‌، کافیست بجای
$A$
قرار دهیم
$A^{-1}$
.
در این صورت
داریم
\begin{equation}
  \begin{split}
    A V^{(i)} = \tilde{V}^{(i - 1)} \\
    \tilde{V}^{(i)} = V^{(i)} / || V^{(i)} ||_\infty
  \end{split}
\end{equation}

که نیازمند حل معادله
$A V^{(i)} = \tilde{V}^{(i - 1)}$
خواهد بود.

حال کافیست ماتریس همراه چندجمله‌ای را در این الگوریتم قرار بدهیم تا بزرگترین یا کوچکترین ریشه‌ی‌آن را بیابیم.
واضح است که پیچیدگی پردازشی این روش،
$O(kn^2)$
خواهد بود که برای
$n$
های بزرگ بهینه تر از روش نیوتون است.

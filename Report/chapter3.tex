\chapter{پیاده‌سازی و پیچیدگی محاسباتی}

\section{پیاده‌سازی روش ماتریس همراه در پایتون}

در این بخش می‌خواهیم به نحوی‌ی پیاده سازی این الگوریتم در پایتون بپردازیم.
فرض کنید چند جمله‌ای دلخواه
$p(x) = a_n x^n + a_{n-1} x^{n-1} + ... + a_1x + a_0$
داده شده است.
در مرحله‌ی اول، باید تمامی ضرایب چند جمله ای را بر
$a_n$
تقسیم کنیم، تا این چند جمله‌ای به فرم مورد نظر ما در آید.
برای ورودی گرفتی یک چند جمله‌ای، کافیست یک آرایه
$n + 1$
تایی داشته باشیم که اندیس
$i$
ام آن متناظر با
$a_{n-i}$
است.

\begin{latin}
  \begin{python}
def get_formated_poly(poly):
    """
    Return the polynomial such that the coefficient of the maximum power of x
    is always 1
    """
    ret = poly / poly[0]
    return ret

  \end{python}
\end{latin}

در مرحله‌ی بعد، باید ماتریس همراه این چندجمله‌ای را بسازیم:
\begin{latin}
  \begin{python}
def get_companion_matrix(poly):
    """
    Calculate the companion matrix for a normalized polynomial
    """
    n = len(poly)
    cmat = np.zeros((n - 1, n - 1))
    cmat[:, n - 2] = -poly[1:]
    cmat[np.arange(1, n - 1), np.arange(0, n - 2)] = 1

    return cmat

  \end{python}
\end{latin}

در آخر، باید مقادیر ویژه‌ی این ماتریس را حساب کنیم. به این منظور می‌توانیم از بخش جبر خطی کتابخوانه
\lr{numpy}
استفاده کنیم یا الگوریتم
\lr{QR}
را پیاده سازی کنیم:

\begin{latin}
  \begin{python}
poly = np.array([2, 3, 1, 4, 3, 6])
norm_poly = get_formated_poly(poly)

eigenvalues, _ = np.linalg.eig(matrix)

  \end{python}
\end{latin}

این روش به طور کلی خطای‌ پایینی دارد و اگرچه ممکن است با ورودی‌های خواص، میزان خطا افزایش یابد،
این میزان در طول محاسبات برای
\lr{QR}
و دترمینان ماتریس نسبتا ثابت می‌مانند~\cite{edelman1995polynomial}

\section{محاسبه حدود ریشه‌ها}

در این بخش، حلقه‌های گرشگورین رو برای ماتریس‌همراه محاسبه‌می‌کنیم.
به این منظور، پس از نرمال کردن چندجمله‌ای و به دست آوردن ماتریس همراه، کافیست بزرگترین مقدار بین
$|a_0|$
تا
$|a_{n-2}|$
را به دست آورده(
فرض کنید
$a_k$
بزرگترین
است
)، و با
$a_{n-1} \pm 1$
مقایسه کنیم
.
ریشه‌های معادله در یکی از
\begin{equation}
  \begin{cases}
    | \lambda | \le a_k  \\
    | \lambda + a_{n-1} | \le 1
  \end{cases}
\end{equation}

قرار خواهند داشت.



\begin{latin}
  \begin{python}

    bounds = [
      (-np.max(np.abs(norm_poly[2:])), +np.max(np.abs(norm_poly[2:]))),
      (-norm_poly[1] - 1, -norm_poly[1] + 1),
    ]

  \end{python}
\end{latin}


\insertfig{../output/gershgorin.pgf}{\lr{Gershgorin Discs for a Companion Matrix}}{GERSHGORIN_CMP_MAT}

\begin{latin}
  \begin{python}

import os
import numpy as np
import matplotlib
import matplotlib.pyplot as plt


def plot_gershgorin_discs(matrix):
    n = len(matrix)
    eigenvalues, _ = np.linalg.eig(matrix)

    fig, ax = plt.subplots()
    ax.set_aspect('equal', adjustable='datalim')

    for i in range(n):
        disc_center = matrix[i, i]
        disc_radius = np.sum(np.abs(matrix[i, :])) - np.abs(matrix[i, i])

        disc = plt.Circle((disc_center, 0), disc_radius, fill=False, color='b', linestyle='dashed')
        ax.add_patch(disc)
        ax.plot(disc_center, 0, 'bo')  # Plot the center of the disc

    ax.set_title('Gershgorin Discs')
    ax.grid(True)
    plt.xlabel('Real')
    plt.ylabel('Imaginary')

    ax.plot(np.real(eigenvalues), np.imag(eigenvalues), 'ro', label='Eigenvalues')
    plt.legend()

    plt.savefig(os.path.join("../output", "gershgorin.jpg"))
    matplotlib.rcParams.update({
        "pgf.texsystem": "xelatex",
        'text.usetex': True,
        'pgf.rcfonts': False,
        "font.family": "mononoki Nerd Font Mono",
        "font.serif": [],
        #  "font.cursive": ["mononoki Nerd Font", "mononoki Nerd Font Mono"],
    })
    plt.savefig(os.path.join("../output", "gershgorin.pgf"))

    plt.show()


def get_companion_matrix(poly):
    """
    Calculate the companion matrix for a normalized polynomial
    """
    n = len(poly)
    cmat = np.zeros((n - 1, n - 1))
    cmat[:, n - 2] = (-poly[1:])[::-1]
    cmat[np.arange(1, n - 1), np.arange(0, n - 2)] = 1

    return cmat


def get_formated_poly(poly):
    """
    Return the polynomial such that the coefficient of the maximum power of x
    is always 1
    """
    return poly / poly[0]


poly = np.array([2, 3, 1, 4, 3, 6])

norm_poly = get_formated_poly(poly)
matrix = get_companion_matrix(norm_poly)

plot_gershgorin_discs(matrix)

  \end{python}
\end{latin}


\section{محاسبه‌ی بزرگترین ریشه‌ی چندجمله‌ای از نظر اندازه}
به این منظور، کافیست الگوریتم توصیف شده در فصل قبل را پیاده سازی کنیم:


\begin{latin}
  \begin{python}

def power_iteration(A, num_iterations=1000, tol=1e-6):
    """
    Power iteration method for finding the dominant eigenvalue and eigenvector.

    Parameters:
    - A: Square matrix for which eigenvalues are to be calculated.
    - num_iterations: Maximum number of iterations (default: 1000).
    - tol: Tolerance to determine convergence (default: 1e-6).

    Returns:
    - eigenvalue: Dominant eigenvalue.
    - eigenvector: Corresponding eigenvector.
    """
    n = A.shape[0]

    # Initialize a random vector
    v = np.random.rand(n)
    v = v / np.linalg.norm(v)  # Normalize the vector

    for i in range(num_iterations):
        Av = np.dot(A, v)
        eigenvalue = np.dot(v, Av)
        v = Av / np.linalg.norm(Av)

        # Check for convergence
        if np.abs(np.dot(Av, v) - eigenvalue) < tol:
            break

    return eigenvalue, v
  \end{python}
\end{latin}



\begin{latin}
  \begin{python}

poly = np.array([2, 3, 1, 4, 3, 6])
norm_poly = get_formated_poly(poly)
matrix = get_companion_matrix(norm_poly)
root, _ = power_iteration(matrix)


  \end{python}
\end{latin}

برای تست نتیجه، می‌توانیم مقدار چندجمله‌ای در این نقطه‌را به دست آوریم:


\begin{latin}
  \begin{python}

def eval_poly(poly, x):
    cur_x = 1
    total = 0
    for a in poly[::-1]:
        total += cur_x * a
        cur_x *= x
    return total
eval_poly(poly, root)

  \end{python}
\end{latin}

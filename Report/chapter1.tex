\chapter{شرح مسئله}

\section{مقدمه}
هدف این پروژه، بررسی روش یافتن ریشه‌های یک چند جمله‌ای با استفاده از ماتریس همراه
\footnote{
\lr{Companion Matrix}
}
آن
و گسترش دادن این روش برای مسائل دیگر است.
در ابتدا، روش‌های مختلف یافتن ریشه‌های چند‌جمله‌ای را بررسی کرده و سپس روش ماترسی همراه را معرفی و اثبات می‌کنیم.
در این راستا، خواص این ماتریس  و ارتباط ریشه‌های چندجمله‌ای با مقادیر ویژه‌ی ماتریس را تحلیل می‌کنیم.

در ادامه، به روش‌های محاسباتی این ماتریس و مقایسه‌ی آن با دیگر روش‌های می‌پردازیم.
دقت و صحت آن را در محاسبات مختلف سنجیده و نقاط قوت و ضعف آن را بررسی می‌کنیم
و کاربرد عملی آن را پیاده سازی می‌کنیم.

با توجه به بررسی نقاط قوت این روش، چند نمونه‌ی عملی از کاربرد‌های آن را می‌پردازیم و استفاده‌های آن در حوزه‌های مختلف را شرح می‌دهیم.

در آخر، به کاربرد ماتریس همراه در حوزه‌های دیگر می‌پردازیم،
از روش مشابهی،‌ برای حل معادلات دیفرانسیل و معادلات مثلثاتی استفاده می‌کنیم و کاربرد های گسترده‌تر این روش رو بررسی می‌کنیم.

برای مشاهده‌ی کد‌های استفاده شده در این گزارش می‌توانید به
\href{https://github.com/atrin-hojjat/PolyRootsWithCompanionMatrix}{اینجا}
مراجعه کنید.

\section{شرح مسئله}
پیدا کردن ریشه‌های یک چند جمله‌ای، از مسائل پر‌کاربرد در حوزه‌‌های مختلف علم است.
چند‌جمله‌ای‌های بستری منعطف برای مدل کردن رابطه‌ی بین متغیر‌ها، توصیف فرایند‌های پیچیده و پیش‌بینی نتایج فراهم می‌کنند.
بررسی رفتار و ریشه‌های چندجمله‌ای ها فهم عمیقی از رفتار سیستم‌های و رابطه‌ی میان عوامل مختلف ارائه می‌دهد.


در علوم کامپیوتر و
\lr{CAD}
،
می‌توان از ریشه‌های چند جمله‌ای برای
\lr{Curve Fitting}
استفاده کرد که لازمه‌ی واقعی‌دیده شدن منحنی‌ها در گرافیک است.
به منظور
تصحیح خطا
\footnote{\lr{ٍError Correcting}}
در
رمزنگاری
\footnote{\lr{Cryptography}}
برای
\lr{encode}
و
\lr{decode}
کردن اطلاعات از چند‌جمله‌ای ها استفاده می‌شود تا از صحت اطلاعت اطمینان برقرار شود و خطا‌های بوجود آمده در زخیره‌سازی و جابجایی اطلاعات تصحیح شود
\cite{FUJIWARA1990171}
.
همجنین بساری از الگوریتم‌های کد‌گذاری، از چندجمله‌ای‌ها در همنهشتی اعداد اول استفاده می‌کنند و پیدا کردن ریشه‌های این چند‌جمله‌ای ها میتواند به شکستن این کدگذاری‌ها منجر شود.
از دیگر کاربرد‌های آن، می‌توان به مدل‌های اقتصادی، طراحی فیلتر در
\lr{Signal Processing}
،
رگرسیون
\footnote{\lr{Regression}}
در آمار
و...
اشاره کرد.

یک چند جمله‌ای بطور کلی، به فرم:

\begin{equation}
  p(x) = a_n x^n + a_{n-1}x^{n-1} + \cdots + a_1x + a_0
\end{equation}

می‌باشد. در این گزارش برای ساده‌سازی، از فرم زیر استفاده می‌کنیم:

\begin{equation}
  p(x) =  x^n + \frac{a_{n-1}}{a_n} x^{n-1} + \cdots + \frac{a_1}{a_n} x + \frac{a_0}{a_n} = x^n + b_{n-1} x^{n-1} + \cdots + b_1 x + b_0
\end{equation}


فرض کنید
$x_1, x_2, \cdots, x_n$
ریشه های چندجمله ای باشند، یعنی:

\begin{equation}
  \forall 1 \le i \le n : p(x_i) = 0
\end{equation}
\begin{equation}
  p(x) = (x - x_1) (x - x_2) \cdots (x - x_n)
\end{equation}

هدف پیدا کردن
$x_i$
هاست.
می‌دانیم هر چندجمله‌آی از درجه‌ی
$n$
،
دقیقا
$n$
ریشه (حقیقی یا موهومی) دارد
در ابتدا می‌خواهیم چند روش کلی برای حل این مسئله را به همراه پیچیدگی محاثباتی آنها بررسی کنیم.

\section{روش‌های بدست آوردن ریشه‌های چندجمله‌ای}
\subsection{روش نیوتون}
روش
نیتون
\footnote{\lr{Newton's Method}}
،
یکی از روش‌های پرکاربرد برای یافتن یک ریشه از چند جمله‌ای‌ست.
در این روش، با شروع از
$x_0$
دلخواه، در هر گام

\begin{equation}
  x_{n + 1} = x_n - p(x_n) / p'(x_n)
\end{equation}
\begin{equation}
  p'(x) = n x^{n-1} + (n - 1) a_{n-1} x^{n-2} + \cdots + a_1
\end{equation}

این روش معمولا در
$O(n^2)$
مرحله ریشه‌ای برای چندجمله‌ای پیدا می‌کند.
اما اگر چند‌جمله‌ای ریشه‌ی حقیقی نداشته باشد و
$x_0$
حقیقی باشد، الگوریتم هیچ‌گاه به یک ریشه‌ی چند جمله‌ای نمیرسد.
اگر
$x_0$
بزرگ تر از بزرگترین ریشه چندجمله‌ای باشد، این الگوریتم در
$O(n^2)$
مرحله به پایان می‌رسد
و بزرگترین ریشه‌ی چندجمله‌ای را پیادا می‌کند.

هر مرحله ای این الگوریتم، نیازمند به‌دست آوردن مقدار چندجمله‌ای و مشتق آن است
که به پیچیدگی محاثباتی
$O(n^4)$
یا
$O(n^3 \log n)$
منجر می‌شود که با استفاده از
\lr{Preprocessing}
یا
\lr{Dynamic Programming}
می‌توان آن را به
$O(n^3)$
کاهش داد.

\subsection{روش هرنر}
در این روش، چند جمله‌ای
$p$
را به فرم زیر بازنویسی می‌کنیم:

\begin{equation}
  p(x) = a_0 + x (a_1 + x (a_2 + x (\cdots +x(a_{n-1} + x a_{n}) \cdots)))
\end{equation}

برای محاسبه‌ی این عبارت، میتوان از سری زیر استفاده کرد

\begin{equation}
  \begin{split}
    b_n = a_n \\
    b_{n-1} = a_{n-1} + b_n x_0 \\
    \vdots \\
    b_1 = a_1 + b_2 x_0 \\
    b_0 = a_0 + b_1 x_0 \\
  \end{split}
\end{equation}

که در آن
$p(x_0)=b_0$
.


فرض کنید چند‌جمله‌ای دارای ریشه‌های

\begin{equation}
  z_n < z_{n-1} < \cdots < z_1
\end{equation}

باشد.
روش
هرنر
\footnote{\lr{Horner's Method}}
در ابتدا با یک حدس
$z_1 < x_0$
شروع می‌کند.
سپس
با روش نیوتون،‌ ریشه
$z_1$
را می‌یابیم.
سپس
$p(x) / (x - z_1)$
را حساب کرده،‌ و با شروع از
$z_1$
همین روند را برای
$z_2$
تا
$z_n$
تکرار می‌کنیم.
حال می‌توان نشان داد

\begin{equation}
  p(x) = (b_1 + b_2x + b_3 x^2 + \cdots + b_{n-1} x^{n - 2} + b_{n} x^{n - 1}) (x - x_0) + b_0
\end{equation}

که برای
$x_0=z_1$

\begin{equation}
  b_0 = p(x_0) = 0
  \implies
  p(x) / (x - x_0) = b_1 + b_2x + b_3 x^2 + \cdots + b_{n-1} x^{n - 2} + b_{n} x^{n - 1}
\end{equation}


\subsection{
  روش دورند-کرنر
}

فرض کنید چند جمله‌ای درجه ۳ پایین را داریم:
\footnote{\lr{Durand-Kerner method}}

\begin{equation}
  p(x) = x^3 + a_2x^2 + a_1x + a_0
\end{equation}

اگر
$A$
،
$B$
و
$C$
ریشه‌های این چند جمله‌ای باشند، داریم:

\begin{equation}
  p(x) = (x - A) (x - B) (x - C)
\end{equation}

\begin{equation}
  A = x - \frac{p(x)}{(x - B) (x - C)}
\end{equation}

برای
$x_0 \ne B, C$

\begin{equation}
  x_1 = x_0 - \frac{p(x_0)}{(x_0 - B) (x_0 - C)}
\end{equation}

این عملیات در یک مرحله
$P$
را به دست می‌آورد.
حال با حدس‌های اولیه
$a_0, b_0, c_0$
می‌توانیم این محاسبات را تکرار کنیم تا مقادیر
$A, B, C$
به دست آیند.
فقط حدس های اولیه باید غیر حقیقی باشند و ریشه ۱ نباشند

\begin{equation}
  \begin{split}
    a_{k + 1} = a_k - \frac{p(a_k)}{(a_k - b_k) (a_k - c_k)} \\
    b_{k + 1} = b_k - \frac{p(b_k)}{(b_k - a_k) (b_k - c_k)} \\
    c_{k + 1} = c_k - \frac{p(c_k)}{(c_k - a_k) (c_k - b_k)}
  \end{split}
\end{equation}


این روش‌را برای چند جمله‌ای با درجه‌ی دلخواه می‌توان گسترش داد.

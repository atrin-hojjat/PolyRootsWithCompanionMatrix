%%%  کلاس AUTthesis، نسخه آبان 1397
%%%   دانشگاه صنعتی امیرکبیر                 http://www.aut.ac.ir
%%%  تالار گفتگوی پارسی‌لاتک،       http://forum.parsilatex.com
%%%   آپدیت شده در آبان 95
%%%   پشتیبانی و راهنمایی          badali_farhad@yahoo.com
%%%
%%%   بازبینی و اصلاح شده در آبان ماه 1397
%%%  Tested via TeXstudio in TeXlive 2014-2018.
%%%

%-----------------------------------------------------------------------------------------------------
%        روش اجرا.: 2 بار F1 ، 2 بار  F11(به منظور تولید مراجع) ، دوبار Ctrl+Alt+I (به منظور تولید نمایه) و دو بار F1 -------> مشاهده Pdf
%%%%%%%%%%%%%%%%%%%%%%%%%%%%%%%%%%%%%%%%%%%%%%%%%%%%%%
%   TeXstudio as your IDE
%%  برای compile در TeXstudio تنها کافی است منوی Options->Configure TeXstudio را زده و در پنجره Configure TeXstudio در بخش Build گزینه Default Compiler را به XeLaTeX تغییر دهید. سند شما به راحتی compile خواهد شد.
%   F1 & F5 : Build & view
%   F6      : Compile
%   F7      : Viewira
%   --------------
%%%%%%%%%%%%%%%%%%%%%%%%%%%%%%%%%%%%%%%%%%%%%%%%%%%%%%
%        اگر قصد نوشتن رساله دکتری را دارید، در خط زیر به جای msc،
%      کلمه phd را قرار دهید. کلیه تنظیمات لازم، به طور خودکار، اعمال می‌شود.
%%% !TEX TS-program = XeLaTeX

\documentclass[oneside,msc,12pt]{AUTthesis}
\usepackage{pgf}

\usepackage{pythonhighlight}

\usepackage{listings}
\usepackage{color}
\usepackage{enumitem}
\usepackage{amsthm}

\definecolor{dkgreen}{rgb}{0,0.6,0}
\definecolor{gray}{rgb}{0.5,0.5,0.5}
\definecolor{mauve}{rgb}{0.58,0,0.82}

\lstset{frame=tb,
  language=Python,
  aboveskip=3mm,
  belowskip=3mm,
  showstringspaces=false,
  columns=flexible,
  basicstyle={\small\ttfamily},
  numbers=none,
  numberstyle=\tiny\color{gray},
  keywordstyle=\color{blue},
  commentstyle=\color{dkgreen},
  stringstyle=\color{mauve},
  breaklines=true,
  breakatwhitespace=true,
  tabsize=3
}


%       فایل commands.tex را حتماً به دقت مطالعه کنید؛ چون دستورات مربوط به فراخوانی بسته زی‌پرشین 
%       و دیگر بسته‌ها و ... در این فایل قرار دارد و بهتر است که با نحوه استفاده از آنها آشنا شوید. توجه شود برای نسخه نهایی پایان‌نامه حتماً hyperref را 
%        غیرفعال کنید.


% در این فایل، دستورها و تنظیمات مورد نیاز، آورده شده است.
%-------------------------------------------------------------------------------------------------------------------
% در ورژن جدید زی‌پرشین برای تایپ متن‌های ریاضی، این سه بسته، حتماً باید فراخوانی شود.
\usepackage{amsthm,amssymb,amsmath,amsfonts}
% بسته‌ای برای تنطیم حاشیه‌های بالا، پایین، چپ و راست صفحه
\usepackage[top=30mm, bottom=30mm, left=25mm, right=30mm]{geometry}
% بسته‌‌ای برای ظاهر شدن شکل‌ها و تصاویر متن
\usepackage{graphicx}
\usepackage{color}
%بسته‌ای برای تنظیم فاصله عمودی خط‌های متن
\usepackage{setspace}
\usepackage{titletoc}
\usepackage{tocloft}
%با فعال کردن بسته زیر فوت‌نوت‌ها در هر صفحه ریست می‌شوند. حالت پیش‌فرض آن ریست شدن در هر فصل می‌باشد.
%\usepackage[perpage]{footmisc}
\usepackage{enumitem}
%\usepackage{titlesec}
% بسته‌ و دستوراتی برای ایجاد لینک‌های رنگی با امکان جهش
\usepackage[pagebackref=false,colorlinks,linkcolor=blue,citecolor=red]{hyperref}
\usepackage[nameinlink]{cleveref}%capitalize,,noabbrev
 \AtBeginDocument{%
    \crefname{equation}{برابری}{equations}%
    \crefname{chapter}{فصل}{chapters}%
    \crefname{section}{بخش}{sections}%
    \crefname{appendix}{پیوست}{appendices}%
    \crefname{enumi}{مورد}{items}%
    \crefname{footnote}{زیرنویس}{footnotes}%
    \crefname{figure}{شکل}{figures}%
    \crefname{table}{جدول}{tables}%
    \crefname{theorem}{قضیه}{theorems}%
    \crefname{lemma}{لم}{lemmas}%
    \crefname{corollary}{نتیجه}{corollaries}%
    \crefname{proposition}{گزاره}{propositions}%
    \crefname{definition}{تعریف}{definitions}%
    \crefname{result}{نتیجه}{results}%
    \crefname{example}{مثال}{examples}%
    \crefname{remark}{نکته}{remarks}%
    \crefname{note}{یادداشت}{notes}%
}
% چنانچه قصد پرینت گرفتن نوشته خود را دارید، خط بالا را غیرفعال و  از دستور زیر استفاده کنید چون در صورت استفاده از دستور زیر‌‌، 
% لینک‌ها به رنگ سیاه ظاهر خواهند شد که برای پرینت گرفتن، مناسب‌تر است
%\usepackage[pagebackref=false]{hyperref}
% بسته‌ لازم برای تنظیم سربرگ‌ها
\usepackage{fancyhdr}
% بسته‌ای برای ظاهر شدن «مراجع»  در فهرست مطالب
\usepackage[nottoc]{tocbibind}
% دستورات مربوط به ایجاد نمایه
\usepackage{makeidx,multicol}
\setlength{\columnsep}{1.5cm}

%%%%%%%%%%%%%%%%%%%%%%%%%%
\usepackage{verbatim}
\makeindex
\usepackage{sectsty}
% فراخوانی بسته زی‌پرشین و تعریف قلم فارسی و انگلیسی
\usepackage{xepersian}%[extrafootnotefeatures]
\SepMark{-}
%حتماً از تک لایو 2014 استفاده کنید.
\settextfont[Scale=1.2]{B Nazanin}
% \setlatintextfont{Times New Roman}
% \setsansfont{Trebuchet MS}
\setmonofont{Mononoki Nerd Font Mono}
\renewcommand{\labelitemi}{$\bullet$}
%%%%%%%%%%%%%%%%%%%%%%%%%%
% چنانچه می‌خواهید اعداد در فرمول‌ها، انگلیسی باشد، خط زیر را غیرفعال کنید.
%در غیر اینصورت حتماً فونت PGaramond را نصب کنید.
%\setdigitfont[Scale=1.1]{PGaramond}%%Yas
%%%%%%%%%%%%%%%%%%%%%%%%%%
% تعریف قلم‌های فارسی اضافی برای استفاده در بعضی از قسمت‌های متن
\defpersianfont\nastaliq[Scale=2]{IranNastaliq}
\defpersianfont\chapternumber[Scale=3]{B Nazanin}
%\chapterfont{\centering}%
%%%%%%%%%%%%%%%%%%%%%%%%%%
% دستوری برای تغییر نام کلمه «اثبات» به «برهان»
\renewcommand\proofname{\textbf{برهان}}

% دستوری برای تغییر نام کلمه «کتاب‌نامه» به «منابع و مراجع«
\renewcommand{\bibname}{منابع و مراجع}


% Headings for every page of ToC, LoF and Lot
\setlength{\cftbeforetoctitleskip}{-1.2em}
\setlength{\cftbeforelottitleskip}{-1.2em}
\setlength{\cftbeforeloftitleskip}{-1.2em}
\setlength{\cftaftertoctitleskip}{-1em}
\setlength{\cftafterlottitleskip}{-1em}
\setlength{\cftafterloftitleskip}{-1em}
%%\makeatletter
%%%%\renewcommand{\l@chapter}{\@dottedtocline{1}{1em\bfseries}{1em}}
%%%%\renewcommand{\l@section}{\@dottedtocline{2}{2em}{2em}}
%%%%\renewcommand{\l@subsection}{\@dottedtocline{3}{3em}{3em}}
%%%%\renewcommand{\l@subsubsection}{\@dottedtocline{4}{4em}{4em}}
%%%%\makeatother


\newcommand\tocheading{\par عنوان\hfill صفحه \par}
\newcommand\lofheading{\hspace*{.5cm}\figurename\hfill صفحه \par}
\newcommand\lotheading{\hspace*{.5cm}\tablename\hfill صفحه \par}

\renewcommand{\cftchapleader}{\cftdotfill{\cftdotsep}}
\renewcommand{\cfttoctitlefont}{\hspace*{\fill}\LARGE\bfseries}%\Large
\renewcommand{\cftaftertoctitle}{\hspace*{\fill}}
\renewcommand{\cftlottitlefont}{\hspace*{\fill}\LARGE\bfseries}%\Large
\renewcommand{\cftafterlottitle}{\hspace*{\fill}}
\renewcommand{\cftloftitlefont}{\hspace*{\fill}\LARGE\bfseries}
\renewcommand{\cftafterloftitle}{\hspace*{\fill}}

%%%%%%%%%%%%%%%%%%%%%%%%%%
% تعریف و نحوه ظاهر شدن عنوان قضیه‌ها، تعریف‌ها، مثال‌ها و ...
%برای شماره گذاری سه تایی قضیه ها
\theoremstyle{definition}
\newtheorem{definition}{تعریف}[section]
\newtheorem{remark}[definition]{نکته}
\newtheorem{note}[definition]{یادداشت}
\newtheorem{example}[definition]{نمونه}
\newtheorem{question}[definition]{سوال}
\newtheorem{remember}[definition]{یاداوری}
\theoremstyle{theorem}
\newtheorem{theorem}[definition]{قضیه}
\newtheorem{lemma}[definition]{لم}
\newtheorem{proposition}[definition]{گزاره}
\newtheorem{corollary}[definition]{نتیجه}
%%%%%%%%%%%%%%%%%%%%%%%%
%%%%%%%%%%%%%%%%%%%
%%% برای شماره گذاری چهارتایی قضیه ها و ...
%%\newtheorem{definition1}[subsubsection]{تعریف}
%%\newtheorem{theorem1}[subsubsection]{قضیه}
%%\newtheorem{lemma1}[subsubsection]{لم}
%%\newtheorem{proposition1}[subsubsection]{گزاره}
%%\newtheorem{corollary1}[subsubsection]{نتیجه}
%%\newtheorem{remark1}[subsubsection]{نکته}
%%\newtheorem{example1}[subsubsection]{مثال}
%%\newtheorem{question1}[subsubsection]{سوال}

%%%%%%%%%%%%%%%%%%%%%%%%%%%%

% دستورهایی برای سفارشی کردن صفحات اول فصل‌ها
\makeatletter
\newcommand\mycustomraggedright{%
 \if@RTL\raggedleft%
 \else\raggedright%
 \fi}
\def\@makechapterhead#1{%
\thispagestyle{style1}
\vspace*{20\p@}%
{\parindent \z@ \mycustomraggedright
\ifnum \c@secnumdepth >\m@ne
\if@mainmatter

\bfseries{\Huge \@chapapp}\small\space {\chapternumber\thechapter}
\par\nobreak
\vskip 0\p@
\fi
\fi
\interlinepenalty\@M 
\Huge \bfseries #1\par\nobreak
\vskip 120\p@

}

%\thispagestyle{empty}
\newpage}
\bidi@patchcmd{\@makechapterhead}{\thechapter}{\tartibi{chapter}}{}{}
\bidi@patchcmd{\chaptermark}{\thechapter}{\tartibi{chapter}}{}{}
\makeatother

\pagestyle{fancy}
\renewcommand{\chaptermark}[1]{\markboth{\chaptername~\tartibi{chapter}: #1}{}}

\fancypagestyle{style1}{
\fancyhf{} 
\fancyfoot[c]{\thepage}
\fancyhead[R]{\leftmark}%
\renewcommand{\headrulewidth}{1.2pt}
}


\fancypagestyle{style2}{
\fancyhf{}
\fancyhead[R]{چکیده}
\fancyfoot[C]{\thepage{}}
\renewcommand{\headrulewidth}{1.2pt}
}

\fancypagestyle{style3}{%
  \fancyhf{}%
  \fancyhead[R]{فهرست نمادها}
  \fancyfoot[C]{\thepage}%
  \renewcommand{\headrulewidth}{1.2pt}%
}

\fancypagestyle{style4}{%
  \fancyhf{}%
  \fancyhead[R]{فهرست جداول}
  \fancyfoot[C]{\thepage}%
  \renewcommand{\headrulewidth}{1.2pt}%
}

\fancypagestyle{style5}{%
  \fancyhf{}%
  \fancyhead[R]{فهرست اشکال}
  \fancyfoot[C]{\thepage}%
  \renewcommand{\headrulewidth}{1.2pt}%
}

\fancypagestyle{style6}{%
  \fancyhf{}%
  \fancyhead[R]{فهرست مطالب}
  \fancyfoot[C]{\thepage}%
  \renewcommand{\headrulewidth}{1.2pt}%
}

\fancypagestyle{style7}{%
  \fancyhf{}%
  \fancyhead[R]{نمایه}
  \fancyfoot[C]{\thepage}%
  \renewcommand{\headrulewidth}{1.2pt}%
}

\fancypagestyle{style8}{%
  \fancyhf{}%
  \fancyhead[R]{منابع و مراجع}
  \fancyfoot[C]{\thepage}%
  \renewcommand{\headrulewidth}{1.2pt}%
}
\fancypagestyle{style9}{%
  \fancyhf{}%
  \fancyhead[R]{واژه‌نامه‌ی فارسی به انگلیسی}
  \fancyfoot[C]{\thepage}%
  \renewcommand{\headrulewidth}{1.2pt}%
}
%


%دستور حذف نام لیست تصاویر و لیست جداول از فهرست مطالب
\newcommand*{\BeginNoToc}{%
  \addtocontents{toc}{%
    \edef\protect\SavedTocDepth{\protect\the\protect\value{tocdepth}}%
  }%
  \addtocontents{toc}{%
    \protect\setcounter{tocdepth}{-10}%
  }%
}
\newcommand*{\EndNoToc}{%
  \addtocontents{toc}{%
    \protect\setcounter{tocdepth}{\protect\SavedTocDepth}%
  }%
}
\newcounter{savepage}
\renewcommand{\listfigurename}{فهرست اشکال}
\renewcommand{\listtablename}{فهرست جداول}
%\renewcommand\cftsecleader{\cftdotfill{\cftdotsep}}
%%%%%%%%%%%%%%%%%%%%%%%%%%%%%
%%%%%%%%%%%%%%%%%%%%%%%%%%%%



\newcommand{\insertfig}[3]{%
\begin{figure}%
	\begin{center}%
		\rmfamily%
    \label{#3}%
		\ttfamily%
		\input{#1}%
		\rmfamily%
	\caption{#2}%
	\end{center}%
\end{figure}%
}%
\newcommand\norm[1]{\left\lVert#1\right\rVert}

\begin{document}
\baselineskip=.75cm
\linespread{1.75}
%% -!TEX root = AUTthesis.tex
%%%%%%%%%%%%%%%%%%%%%%%%%%%%%%%%%%%%
\faculty{دانشکده ریاضی و علوم کامپیوتر}
\department{گرایش ریاضی}
\fatitle{یافتن ریشه‌های چندجمله‌ای با ماتریس همراه آن}
% نام استاد(ان) راهنما را وارد کنید
% \firstsupervisor{نام کامل استاد راهنما}
%\secondsupervisor{استاد راهنمای دوم}
% نام استاد(دان) مشاور را وارد کنید. چنانچه استاد مشاور ندارید، دستور پایین را غیرفعال کنید.
% \firstadvisor{نام کامل استاد مشاور}
%\secondadvisor{استاد مشاور دوم}
% نام نویسنده را وارد کنید
\name{آترین}
% نام خانوادگی نویسنده را وارد کنید
\surname{حجت}
%%%%%%%%%%%%%%%%%%%%%%%%%%%%%%%%%%
\thesisdate{بهمن ۱۴۰۲}

% چکیده پایان‌نامه را وارد کنید
\fa-abstract{
}


% کلمات کلیدی پایان‌نامه را وارد کنید
\keywords{جبر خطی عددی}



\AUTtitle
%%%%%%%%%%%%%%%%%%%%%%%%%%%%%%%%%%
\vspace*{7cm}
\thispagestyle{empty}

\pagenumbering{alph}
%{\pagestyle{style2}
%\tableofcontents}\newpage
%
%\listoffigures
\cleardoublepage
\pagestyle{style6}
\tableofcontents
\pagestyle{style6}
\cleardoublepage
%اگر لیست تصاویر و لیست جداول ندارید ، کدهای زیر را با گذاشتن % در ابتدای آنها، غیرفعال کنید.
\BeginNoToc
%============
\addtocontents{lof}{\lofheading}% add heading to the first page in LoF
\pagestyle{style5}
\listoffigures
\thispagestyle{style5}
\cleardoublepage
%============
\addtocontents{lot}{\lotheading}% add heading to the first page in LoT
\thispagestyle{style4}
\listoftables
\thispagestyle{style4}
%============
%\cleardoublepage
%
\cleardoublepage
\setcounter{savepage}{\arabic{page}}
\mainmatter
\addtocontents{toc}{\tocheading}% add heading to the first page in ToC, after frontmatter entries
\EndNoToc
%%%%%%%%%%%%%

{\centering\LARGE\textbf{فهرست نمادها}\par}%

\pagenumbering{alph}
\setcounter{page}{\thesavepage}
%\setcounter{page}{6}
\vspace*{1cm}

\pagestyle{style3}
%\thispagestyle{empty}
%\addcontentsline{toc}{chapter}{فهرست نمادها}
\symb{\text{ نماد}}{مفهوم}
\\
%مقادیر بالا را تغییر ندهید
%%%%%%%%%%%%%%%%%%%%%%%%%%%%%%%%%%%%%%%%%%%%%%%%%%%%%%%%%
\symb{p(x)}{
چند جمله‌ای بر پایه‌ی $x$
}
%%%%%%%%%%%%%%%%%%%%%%%%%%%%%%%%%%%%%%%

\thispagestyle{style3}
\newpage
%\pagestyle{style1}
%%%%%%%%%%%%%%%%%%%%%%%%%%%%%%%%%%%%


\pagenumbering{arabic}
\pagestyle{style1}
%--------------------------------------------------------------------------chapters(فصل ها)
\chapter{شرح مسئله}

\section{مقدمه}
هدف این پروژه، بررسی روش یافتن ریشه‌های یک چند جمله‌ای با استفاده از ماتریس همراه
\footnote{
\lr{Companion Matrix}
}
آن
و گسترش دادن این روش برای مسائل دیگر است.
در ابتدا، روش‌های مختلف یافتن ریشه‌های چند‌جمله‌ای را بررسی کرده و سپس روش ماترسی همراه را معرفی و اثبات می‌کنیم.
در این راستا، خواص این ماتریس  و ارتباط ریشه‌های چندجمله‌ای با مقادیر ویژه‌ی ماتریس را تحلیل می‌کنیم.

در ادامه، به روش‌های محاسباتی این ماتریس و مقایسه‌ی آن با دیگر روش‌های می‌پردازیم.
دقت و صحت آن را در محاسبات مختلف سنجیده و نقاط قوت و ضعف آن را بررسی می‌کنیم
و کاربرد عملی آن را پیاده سازی می‌کنیم.

با توجه به بررسی نقاط قوت این روش، چند نمونه‌ی عملی از کاربرد‌های آن را می‌پردازیم و استفاده‌های آن در حوزه‌های مختلف را شرح می‌دهیم.

در آخر، به کاربرد ماتریس همراه در حوزه‌های دیگر می‌پردازیم،
از روش مشابهی،‌ برای حل معادلات دیفرانسیل و معادلات مثلثاتی استفاده می‌کنیم و کاربرد های گسترده‌تر این روش رو بررسی می‌کنیم.

برای مشاهده‌ی کد‌های استفاده شده در این گزارش می‌توانید به
\href{https://github.com/atrin-hojjat/PolyRootsWithCompanionMatrix}{اینجا}
مراجعه کنید.

\section{شرح مسئله}
پیدا کردن ریشه‌های یک چند جمله‌ای، از مسائل پر‌کاربرد در حوزه‌‌های مختلف علم است.
چند‌جمله‌ای‌های بستری منعطف برای مدل کردن رابطه‌ی بین متغیر‌ها، توصیف فرایند‌های پیچیده و پیش‌بینی نتایج فراهم می‌کنند.
بررسی رفتار و ریشه‌های چندجمله‌ای ها فهم عمیقی از رفتار سیستم‌های و رابطه‌ی میان عوامل مختلف ارائه می‌دهد.


در علوم کامپیوتر و
\lr{CAD}
،
می‌توان از ریشه‌های چند جمله‌ای برای
\lr{Curve Fitting}
استفاده کرد که لازمه‌ی واقعی‌دیده شدن منحنی‌ها در گرافیک است.
به منظور
تصحیح خطا
\footnote{\lr{ٍError Correcting}}
در
رمزنگاری
\footnote{\lr{Cryptography}}
برای
\lr{encode}
و
\lr{decode}
کردن اطلاعات از چند‌جمله‌ای ها استفاده می‌شود تا از صحت اطلاعت اطمینان برقرار شود و خطا‌های بوجود آمده در زخیره‌سازی و جابجایی اطلاعات تصحیح شود
\cite{FUJIWARA1990171}
.
همجنین بساری از الگوریتم‌های کد‌گذاری، از چندجمله‌ای‌ها در همنهشتی اعداد اول استفاده می‌کنند و پیدا کردن ریشه‌های این چند‌جمله‌ای ها میتواند به شکستن این کدگذاری‌ها منجر شود.
از دیگر کاربرد‌های آن، می‌توان به مدل‌های اقتصادی، طراحی فیلتر در
\lr{Signal Processing}
،
رگرسیون
\footnote{\lr{Regression}}
در آمار
و...
اشاره کرد.

یک چند جمله‌ای بطور کلی، به فرم:

\begin{equation}
  p(x) = a_n x^n + a_{n-1}x^{n-1} + \cdots + a_1x + a_0
\end{equation}

می‌باشد. در این گزارش برای ساده‌سازی، از فرم زیر استفاده می‌کنیم:

\begin{equation}
  p(x) =  x^n + \frac{a_{n-1}}{a_n} x^{n-1} + \cdots + \frac{a_1}{a_n} x + \frac{a_0}{a_n} = x^n + b_{n-1} x^{n-1} + \cdots + b_1 x + b_0
\end{equation}


فرض کنید
$x_1, x_2, \cdots, x_n$
ریشه های چندجمله ای باشند، یعنی:

\begin{equation}
  \forall 1 \le i \le n : p(x_i) = 0
\end{equation}
\begin{equation}
  p(x) = (x - x_1) (x - x_2) \cdots (x - x_n)
\end{equation}

هدف پیدا کردن
$x_i$
هاست.
می‌دانیم هر چندجمله‌آی از درجه‌ی
$n$
،
دقیقا
$n$
ریشه (حقیقی یا موهومی) دارد
در ابتدا می‌خواهیم چند روش کلی برای حل این مسئله را به همراه پیچیدگی محاثباتی آنها بررسی کنیم.

\section{روش‌های بدست آوردن ریشه‌های چندجمله‌ای}
\subsection{روش نیوتون}
روش
نیتون
\footnote{\lr{Newton's Method}}
،
یکی از روش‌های پرکاربرد برای یافتن یک ریشه از چند جمله‌ای‌ست.
در این روش، با شروع از
$x_0$
دلخواه، در هر گام

\begin{equation}
  x_{n + 1} = x_n - p(x_n) / p'(x_n)
\end{equation}
\begin{equation}
  p'(x) = n x^{n-1} + (n - 1) a_{n-1} x^{n-2} + \cdots + a_1
\end{equation}

این روش معمولا در
$O(n^2)$
مرحله ریشه‌ای برای چندجمله‌ای پیدا می‌کند.
اما اگر چند‌جمله‌ای ریشه‌ی حقیقی نداشته باشد و
$x_0$
حقیقی باشد، الگوریتم هیچ‌گاه به یک ریشه‌ی چند جمله‌ای نمیرسد.
اگر
$x_0$
بزرگ تر از بزرگترین ریشه چندجمله‌ای باشد، این الگوریتم در
$O(n^2)$
مرحله به پایان می‌رسد
و بزرگترین ریشه‌ی چندجمله‌ای را پیادا می‌کند.

هر مرحله ای این الگوریتم، نیازمند به‌دست آوردن مقدار چندجمله‌ای و مشتق آن است
که به پیچیدگی محاثباتی
$O(n^4)$
یا
$O(n^3 \log n)$
منجر می‌شود که با استفاده از
\lr{Preprocessing}
یا
\lr{Dynamic Programming}
می‌توان آن را به
$O(n^3)$
کاهش داد.

\subsection{روش هرنر}
در این روش، چند جمله‌ای
$p$
را به فرم زیر بازنویسی می‌کنیم:

\begin{equation}
  p(x) = a_0 + x (a_1 + x (a_2 + x (\cdots +x(a_{n-1} + x a_{n}) \cdots)))
\end{equation}

برای محاسبه‌ی این عبارت، میتوان از سری زیر استفاده کرد

\begin{equation}
  \begin{split}
    b_n = a_n \\
    b_{n-1} = a_{n-1} + b_n x_0 \\
    \vdots \\
    b_1 = a_1 + b_2 x_0 \\
    b_0 = a_0 + b_1 x_0 \\
  \end{split}
\end{equation}

که در آن
$p(x_0)=b_0$
.


فرض کنید چند‌جمله‌ای دارای ریشه‌های

\begin{equation}
  z_n < z_{n-1} < \cdots < z_1
\end{equation}

باشد.
روش
هرنر
\footnote{\lr{Horner's Method}}
در ابتدا با یک حدس
$z_1 < x_0$
شروع می‌کند.
سپس
با روش نیوتون،‌ ریشه
$z_1$
را می‌یابیم.
سپس
$p(x) / (x - z_1)$
را حساب کرده،‌ و با شروع از
$z_1$
همین روند را برای
$z_2$
تا
$z_n$
تکرار می‌کنیم.
حال می‌توان نشان داد

\begin{equation}
  p(x) = (b_1 + b_2x + b_3 x^2 + \cdots + b_{n-1} x^{n - 2} + b_{n} x^{n - 1}) (x - x_0) + b_0
\end{equation}

که برای
$x_0=z_1$

\begin{equation}
  b_0 = p(x_0) = 0
  \implies
  p(x) / (x - x_0) = b_1 + b_2x + b_3 x^2 + \cdots + b_{n-1} x^{n - 2} + b_{n} x^{n - 1}
\end{equation}


\subsection{
  روش دورند-کرنر
}

فرض کنید چند جمله‌ای درجه ۳ پایین را داریم:
\footnote{\lr{Durand-Kerner method}}

\begin{equation}
  p(x) = x^3 + a_2x^2 + a_1x + a_0
\end{equation}

اگر
$A$
،
$B$
و
$C$
ریشه‌های این چند جمله‌ای باشند، داریم:

\begin{equation}
  p(x) = (x - A) (x - B) (x - C)
\end{equation}

\begin{equation}
  A = x - \frac{p(x)}{(x - B) (x - C)}
\end{equation}

برای
$x_0 \ne B, C$

\begin{equation}
  x_1 = x_0 - \frac{p(x_0)}{(x_0 - B) (x_0 - C)}
\end{equation}

این عملیات در یک مرحله
$P$
را به دست می‌آورد.
حال با حدس‌های اولیه
$a_0, b_0, c_0$
می‌توانیم این محاسبات را تکرار کنیم تا مقادیر
$A, B, C$
به دست آیند.
فقط حدس های اولیه باید غیر حقیقی باشند و ریشه ۱ نباشند

\begin{equation}
  \begin{split}
    a_{k + 1} = a_k - \frac{p(a_k)}{(a_k - b_k) (a_k - c_k)} \\
    b_{k + 1} = b_k - \frac{p(b_k)}{(b_k - a_k) (b_k - c_k)} \\
    c_{k + 1} = c_k - \frac{p(c_k)}{(c_k - a_k) (c_k - b_k)}
  \end{split}
\end{equation}


این روش‌را برای چند جمله‌ای با درجه‌ی دلخواه می‌توان گسترش داد.

\chapter{روش ماتریس همراه}

در این فصل، روش ماتریس همراه را معرفی می‌کنیم.
این روش، قابل اعتمار ترین روش برای محاسبه‌ی ریشه‌های تکراری یک چند جمله‌ای ست.
هردو روش بررسی شده در بالا، با فرض عدم وجود ریشه‌های تکراری کار می‌کنند.
محاسبه‌ی این ریشه‌ها در عملیات‌های کامپیوتر معمولا خطای زیادی دارد.
از این رو پیدا کردن روشی که بتواند این ریشه‌ها را نیز با دقت خوبی محاسبه کند بسیار حائز اهمیت است.

\section{بررسی روش ماتریس همراه}
فرض کنید چند جمله‌ای
$p$
به فرم

\begin{equation}
  p(x) = x^n + a_{n-1} x^{n-1} + \cdots + a_1 x + a_0
\end{equation}

داده شده است.
ماتریس همراه آن را به شکل زیر تعریف می‌کنیم:

\begin{equation}
  C =
    \begin{bmatrix}
      0 & 0 & 0 & \cdots & 0 & -a_0 \\
      1 & 0 & 0 & \cdots & 0 & -a_1 \\
      0 & 1 & 0 & \cdots & 0 & -a_2 \\
      0 & 0 & 1 & \cdots & 0 & -a_3 \\
      \vdots &  &  & \ddots & & \vdots \\
      0 & 0 & 0 & \cdots & 1 & -a_{n-1} \\

    \end{bmatrix}
\end{equation}


در این روش، مقادری ویژه‌ی ماتریس
$C$
همان ریشه‌های چندجمله‌ای
$p$
خواهد بود.

این روش سریع‌ترین یا بهینه ترین روش برای محاسبه‌ی ریشه‌های چند جمله‌ای نیست، از آنجایی که
نیاز به
$O(n^2)$
حافظه و
$O(n^3)$
محاسبه دارد، در حالی که یک الگوریتم بهینه برای به دست آوردن ریشه‌های چندجمله‌ای میتواند با حافظه
$O(n)$
و پیچیدگی محاسباتی
$O(n^2)$
کار کند.
از طرفی، هرچه تعداد محاسبات بیشتر باشد، میزان خطای این روش نیز بیشتر خواهد بود~\cite{moler1991cleve}
.

در محاسبه‌ی مقادیر ویژه‌ی یک ماتریس، میزان خطا در کامپیوتر حداقل از اردر
$O(\epsilon ||A||_F)$
که
$\epsilon$
دقت ماشین و
$||A||_F$
نرم فروبنیوس ماتریس است~\cite{osborne1960pre}
.
برای کاهش این خطا، می‌توان از تغییرات قطری
\footnote{\lr{Diagonal similarity transformations}}
استفاده کرد بطوری‌که نرم
$A$
کاهش یابد~\cite{beresford1969}~\cite{james2014matrix}~\cite{osborne1960pre}
.

پس از نرمال کردن ماتریس، می‌توان از الگوریتم
\lr{QR}
برای به دست آوردن مقادیر ویژه‌ی ماتریس استفاده کرد.
در این الگوریتم، در مرحله
$k$
ام،
ابتدا تجزیه‌ی
\lr{QR}
برای ماتریس
$A_k$
به فرم
$A_k = Q_k R_k$
را به
دست می‌آوریم که
$A_0 = A$
.
سپس در هر مرحله
قرار می‌دهیم
$A_{k + 1} = R_k Q_k$
.
با توجه به معدله‌ی زیر، می‌توان دید که در هر مرحله،
$A_k$
مشابه
$A$
است.

\begin{equation}
  A_{k+1} = R_k Q_k = Q_{k}^{-1} Q_k R_k Q_k = Q_k^{-1} A Q_k = Q_k^T A Q_k
\end{equation}

پس از تعدادی مرحله، ماتریس
$A_k$
به ماتریسی مثلثی تبدیل می‌شود که به آن شکل شور
\footnote{\lr{Schor form of $A$}}
$A$
گفته می‌شود.
مقادیر ویژه‌ی ماتریس‌های مثلثی، برابر مقادیر روی قطر آنهاست. با توجه به اینکه
$A$
و
$A_k$
مشابه‌اند، مقادیر ویژه‌ی
$A$
نیز به دست آمده است.
ممکن است شرایطی به وجود آید که
$A_k$
مثلثی نشود~\cite{golub2013matrix}
.


\section{اثبات روش ماتریس همراه}

به منظور اثبات درستی این روش، کافیست نشان دهیم که چندجمله‌ای مشخصه‌ی ماتریس
$C$
به فرم
$det(Ix - C)$
برابر چندجمله‌ای
$p(x)$
است.

برای اثبات صحت این روش، از استقرا استفاده می‌کنیم:

\begin{proof}
  پایه: $\left(n=1\right)$
  $$ A = [a_0], det(Ix - A) = x - a_0 $$

  گام:
  \begin{equation}
    Ix - C =
    \begin{bmatrix}
      x & 0 & 0 & \cdots & 0 & a_0 \\
      -1 & x & 0 & \cdots & 0 & a_1 \\
      0 & -1 & x & \cdots & 0 & a_2 \\
      0 & 0 & -1 & \cdots & 0 & a_3 \\
      \vdots &  &  & \ddots & & \vdots \\
      0 & 0 & 0 & \cdots & -1 & x + a_{n-1} \\
    \end{bmatrix}
  \end{equation}
  \begin{equation}
    \begin{split}
      det(Ix - C) =
      \begin{vmatrix}
        x & 0 & 0 & \cdots & 0 & a_0 \\
        -1 & x & 0 & \cdots & 0 & a_1 \\
        0 & -1 & x & \cdots & 0 & a_2 \\
        0 & 0 & -1 & \cdots & 0 & a_3 \\
        \vdots &  &  & \ddots & & \vdots \\
        0 & 0 & 0 & \cdots & -1 & x + a_{n-1} \\
      \end{vmatrix}
      \\
      =
      x
      \begin{vmatrix}
        x & 0 & \cdots & 0 & a_1 \\
        -1 & x & \cdots & 0 & a_2 \\
        0 & -1 & \cdots & 0 & a_3 \\
        \vdots &  &  & \ddots & & \vdots \\
        0 & 0 & \cdots & -1 & x + a_{n-1} \\
      \end{vmatrix}
      -
      (-1)^{n+1}a_0
      \begin{vmatrix}
        x & 0 & \cdots & 0 \\
        -1 & x & \cdots & 0 \\
        0 & -1 & \cdots & 0 \\
        \vdots &  & \ddots & \vdots \\
        0 & 0 & \cdots & -1 \\
      \end{vmatrix}
    \end{split}

  \end{equation}

  طبق فرض استقرا داریم:

  \begin{equation}
    \begin{vmatrix}
      x & 0 & \cdots & 0 & a_1 \\
      -1 & x & \cdots & 0 & a_2 \\
      0 & -1 & \cdots & 0 & a_3 \\
      \vdots &  &  & \ddots & \vdots \\
      0 & 0 & \cdots & -1 & x + a_{n-1} \\
    \end{vmatrix}
    = x^{n-1} + a_{n-1}x^{n-2} + \cdots + a_2 x + a_1
  \end{equation}
  و دترمینان ماتریس دوم به دلیل مثلثی بودن، برابر با حاصل ضرب مقادیر روی قطر آن است

  \begin{equation}
    \begin{split}
        det(Ix - C) = x (x^{n-1} + a_{n-1}x^{n-2} + \cdots + a_2 x + a_1) + (-1)^{n + 1} a_0 (-1)^{n-1} \\
        = x^n + a_{n-1} x^{n-1} + \cdots + a_1x + a_0
    \end{split}
  \end{equation}

\end{proof}


\section{محدود کردن بازه‌ی ریشه‌های معادله}

همانطور که در بخش ۱ دیدیم، روش‌هایی مانند روش نیوتون نیازمند وجود یک تخمین از محدوده‌ی ریشه‌هاست.
برای مثال، روش نیوتون نیاز داشت که حداکثر مقدار ریشه‌های چندجمله‌ای را داشته باشد، تا حدس اولیه‌اش بزرگ تر از بزرگترین ریشه‌ی چندجمله‌ای باشد.
در این بخش می‌خواهیم استفاده‌ی ماتریس همراه را در تخمین مقادیر ریشه‌ها بررسی کنیم.

در فصل ۵ درس، دیدیم که می‌توان از حلقه‌های گرشگروین
\footnote{\lr{Gershgorin}}
استفاده کرد تا محدوده‌ی مقادیر ویژه‌ی یک ماتریس را به دست آورد.
شعاع هر یک از این
$n$
حلقه از رابطه‌ی
\begin{equation}
  r_i = \sum\limits_{
    \substack{
      j = 1 \\
      i \ne j
    }
  }^{n} |a_{ij}|
\end{equation}
هر یک از مقادیر ویژه‌ی ماتریس
$A$
باید حداقل در یکی از نامساوی‌های زیر صدق کند:
\begin{equation}
  | \lambda - a_{ii} | \le r_i
\end{equation}

از آنجایی که

\begin{equation}
  C =
    \begin{bmatrix}
      0 & 0 & 0 & \cdots & 0 & -a_0 \\
      1 & 0 & 0 & \cdots & 0 & -a_1 \\
      0 & 1 & 0 & \cdots & 0 & -a_2 \\
      0 & 0 & 1 & \cdots & 0 & -a_3 \\
      \vdots &  &  & \ddots & & \vdots \\
      0 & 0 & 0 & \cdots & 1 & -a_{n-1} \\

    \end{bmatrix}
\end{equation}

داریم

\begin{equation}
  \begin{cases}
    | \lambda | \le a_i & \forsome 0 \le i < n - 1 \text{\lr{or}} \\
    | \lambda + a_{n-1} | \le 1 & \text{\lr{else}}.
  \end{cases}
\end{equation}

یعنی یک دیسک با شعاع
$1$
در مرکز
$a_{n-1}$
و
$n-2$
دیسک با مرکز
$0$
قرار می‌گیرند.
پس می‌توان گفت که ریشه‌های یک چندجمله‌ای دی یکی از دو بازه‌ی زیر اند
\begin{equation}
  \begin{cases}
    | \lambda | \le \max_{0 \le i < n - 1} a_i & \text{\lr{or}} \\
    | \lambda + a_{n-1} | \le 1 & \text{\lr{else}}.
  \end{cases}
\end{equation}

\insertfig{../output/gershgorin.pgf}{\lr{Gershgorin Discs for a Companion Matrix}}{GERSHGORIN_CMP_MAT}


\section{پیدا کردن کوچکترین یا بزرگترین ریشه‌ی یک چندجمله‌ای با کمک روش توانی}

می‌دانیم یک روش برای محاسبه‌ی یک مقدار ویژه‌ی ماتریس، استفاده از روش توانی است.
در این ورش، میتوان بصورت تکراری با شروع از یک بردار دلخواه، بزرگترین مقدار ویژه‌ی یک ماتریس را محاسبه‌کرد.

در این روش با شروع از
$V^{(0)}$
در هر مرحله

\begin{equation}
  V^{(i)} = A V^{(i - 1)}
\end{equation}

برای کاهش خطای محاسبه می‌توان قرار داد:
\begin{equation}
  \begin{split}
    V^{(i)} = A \tilde{V}^{(i - 1)} \\
    \tilde{V}^{(i)} = V^{(i)} / || V^{(i)} ||_\infty
  \end{split}
\end{equation}

بطوری که


\begin{equation}
  \begin{split}
    \lambda \approx \frac{\tilde{V}_j^{(i)}}{\tilde{V}_j^{(i - 1)}}
  \end{split}
\end{equation}

برای محاسبه‌ی کوچکترین مقدار ویژه‌، کافیست بجای
$A$
قرار دهیم
$A^{-1}$
.
در این صورت
داریم
\begin{equation}
  \begin{split}
    A V^{(i)} = \tilde{V}^{(i - 1)} \\
    \tilde{V}^{(i)} = V^{(i)} / || V^{(i)} ||_\infty
  \end{split}
\end{equation}

که نیازمند حل معادله
$A V^{(i)} = \tilde{V}^{(i - 1)}$
خواهد بود.

حال کافیست ماتریس همراه چندجمله‌ای را در این الگوریتم قرار بدهیم تا بزرگترین یا کوچکترین ریشه‌ی‌آن را بیابیم.
واضح است که پیچیدگی پردازشی این روش،
$O(kn^2)$
خواهد بود که برای
$n$
های بزرگ بهینه تر از روش نیوتون است.

\chapter{پیاده‌سازی و پیچیدگی محاسباتی}

\section{پیاده‌سازی روش ماتریس همراه در پایتون}

در این بخش می‌خواهیم به نحوی‌ی پیاده سازی این الگوریتم در پایتون بپردازیم.
فرض کنید چند جمله‌ای دلخواه
$p(x) = a_n x^n + a_{n-1} x^{n-1} + ... + a_1x + a_0$
داده شده است.
در مرحله‌ی اول، باید تمامی ضرایب چند جمله ای را بر
$a_n$
تقسیم کنیم، تا این چند جمله‌ای به فرم مورد نظر ما در آید.
برای ورودی گرفتی یک چند جمله‌ای، کافیست یک آرایه
$n + 1$
تایی داشته باشیم که اندیس
$i$
ام آن متناظر با
$a_{n-i}$
است.

\begin{latin}
  \begin{python}
def get_formated_poly(poly):
    """
    Return the polynomial such that the coefficient of the maximum power of x
    is always 1
    """
    ret = poly / poly[0]
    return ret

  \end{python}
\end{latin}

در مرحله‌ی بعد، باید ماتریس همراه این چندجمله‌ای را بسازیم:
\begin{latin}
  \begin{python}
def get_companion_matrix(poly):
    """
    Calculate the companion matrix for a normalized polynomial
    """
    n = len(poly)
    cmat = np.zeros((n - 1, n - 1))
    cmat[:, n - 2] = -poly[1:]
    cmat[np.arange(1, n - 1), np.arange(0, n - 2)] = 1

    return cmat

  \end{python}
\end{latin}

در آخر، باید مقادیر ویژه‌ی این ماتریس را حساب کنیم. به این منظور می‌توانیم از بخش جبر خطی کتابخوانه
\lr{numpy}
استفاده کنیم یا الگوریتم
\lr{QR}
را پیاده سازی کنیم:

\begin{latin}
  \begin{python}
poly = np.array([2, 3, 1, 4, 3, 6])
norm_poly = get_formated_poly(poly)

eigenvalues, _ = np.linalg.eig(matrix)

  \end{python}
\end{latin}

این روش به طور کلی خطای‌ پایینی دارد و اگرچه ممکن است با ورودی‌های خواص، میزان خطا افزایش یابد،
این میزان در طول محاسبات برای
\lr{QR}
و دترمینان ماتریس نسبتا ثابت می‌مانند~\cite{edelman1995polynomial}

\section{محاسبه حدود ریشه‌ها}

در این بخش، حلقه‌های گرشگورین رو برای ماتریس‌همراه محاسبه‌می‌کنیم.
به این منظور، پس از نرمال کردن چندجمله‌ای و به دست آوردن ماتریس همراه، کافیست بزرگترین مقدار بین
$|a_0|$
تا
$|a_{n-2}|$
را به دست آورده(
فرض کنید
$a_k$
بزرگترین
است
)، و با
$a_{n-1} \pm 1$
مقایسه کنیم
.
ریشه‌های معادله در یکی از
\begin{equation}
  \begin{cases}
    | \lambda | \le a_k  \\
    | \lambda + a_{n-1} | \le 1
  \end{cases}
\end{equation}

قرار خواهند داشت.



\begin{latin}
  \begin{python}

    bounds = [
      (-np.max(np.abs(norm_poly[2:])), +np.max(np.abs(norm_poly[2:]))),
      (-norm_poly[1] - 1, -norm_poly[1] + 1),
    ]

  \end{python}
\end{latin}


\insertfig{../output/gershgorin.pgf}{\lr{Gershgorin Discs for a Companion Matrix}}{GERSHGORIN_CMP_MAT}

\begin{latin}
  \begin{python}

import os
import numpy as np
import matplotlib
import matplotlib.pyplot as plt


def plot_gershgorin_discs(matrix):
    n = len(matrix)
    eigenvalues, _ = np.linalg.eig(matrix)

    fig, ax = plt.subplots()
    ax.set_aspect('equal', adjustable='datalim')

    for i in range(n):
        disc_center = matrix[i, i]
        disc_radius = np.sum(np.abs(matrix[i, :])) - np.abs(matrix[i, i])

        disc = plt.Circle((disc_center, 0), disc_radius, fill=False, color='b', linestyle='dashed')
        ax.add_patch(disc)
        ax.plot(disc_center, 0, 'bo')  # Plot the center of the disc

    ax.set_title('Gershgorin Discs')
    ax.grid(True)
    plt.xlabel('Real')
    plt.ylabel('Imaginary')

    ax.plot(np.real(eigenvalues), np.imag(eigenvalues), 'ro', label='Eigenvalues')
    plt.legend()

    plt.savefig(os.path.join("../output", "gershgorin.jpg"))
    matplotlib.rcParams.update({
        "pgf.texsystem": "xelatex",
        'text.usetex': True,
        'pgf.rcfonts': False,
        "font.family": "mononoki Nerd Font Mono",
        "font.serif": [],
        #  "font.cursive": ["mononoki Nerd Font", "mononoki Nerd Font Mono"],
    })
    plt.savefig(os.path.join("../output", "gershgorin.pgf"))

    plt.show()


def get_companion_matrix(poly):
    """
    Calculate the companion matrix for a normalized polynomial
    """
    n = len(poly)
    cmat = np.zeros((n - 1, n - 1))
    cmat[:, n - 2] = (-poly[1:])[::-1]
    cmat[np.arange(1, n - 1), np.arange(0, n - 2)] = 1

    return cmat


def get_formated_poly(poly):
    """
    Return the polynomial such that the coefficient of the maximum power of x
    is always 1
    """
    return poly / poly[0]


poly = np.array([2, 3, 1, 4, 3, 6])

norm_poly = get_formated_poly(poly)
matrix = get_companion_matrix(norm_poly)

plot_gershgorin_discs(matrix)

  \end{python}
\end{latin}


\section{محاسبه‌ی بزرگترین ریشه‌ی چندجمله‌ای از نظر اندازه}
به این منظور، کافیست الگوریتم توصیف شده در فصل قبل را پیاده سازی کنیم:


\begin{latin}
  \begin{python}

def power_iteration(A, num_iterations=1000, tol=1e-6):
    """
    Power iteration method for finding the dominant eigenvalue and eigenvector.

    Parameters:
    - A: Square matrix for which eigenvalues are to be calculated.
    - num_iterations: Maximum number of iterations (default: 1000).
    - tol: Tolerance to determine convergence (default: 1e-6).

    Returns:
    - eigenvalue: Dominant eigenvalue.
    - eigenvector: Corresponding eigenvector.
    """
    n = A.shape[0]

    # Initialize a random vector
    v = np.random.rand(n)
    v = v / np.linalg.norm(v)  # Normalize the vector

    for i in range(num_iterations):
        Av = np.dot(A, v)
        eigenvalue = np.dot(v, Av)
        v = Av / np.linalg.norm(Av)

        # Check for convergence
        if np.abs(np.dot(Av, v) - eigenvalue) < tol:
            break

    return eigenvalue, v
  \end{python}
\end{latin}



\begin{latin}
  \begin{python}

poly = np.array([2, 3, 1, 4, 3, 6])
norm_poly = get_formated_poly(poly)
matrix = get_companion_matrix(norm_poly)
root, _ = power_iteration(matrix)


  \end{python}
\end{latin}

برای تست نتیجه، می‌توانیم مقدار چندجمله‌ای در این نقطه‌را به دست آوریم:


\begin{latin}
  \begin{python}

def eval_poly(poly, x):
    cur_x = 1
    total = 0
    for a in poly[::-1]:
        total += cur_x * a
        cur_x *= x
    return total
eval_poly(poly, root)

  \end{python}
\end{latin}

%\include{chapter4}
%\include{chapter5}

%--------------------------------------------------------------------------appendix( مراجع و پیوست ها)
\chapterfont{\vspace*{-2em}\centering\LARGE}%

\appendix
\bibliographystyle{plain-fa}
\bibliography{references}
\chapter*{‌پیوست}
\markboth{پیوست}{}
\addcontentsline{toc}{chapter}{پیوست}
%--------------------------------------------------------------------------dictionary(واژه نامه ها)
%اگر مایل به داشتن صفحه واژه‌نامه نیستید، خط زیر را غیر فعال کنید.
%\parindent=0pt
%%
\chapter*{واژه‌نامه‌ی فارسی به انگلیسی}
\pagestyle{style9}

\addcontentsline{toc}{chapter}{واژه‌نامه‌ی فارسی به انگلیسی}
%%%%%%
\begin{multicols*}{2}

{\bf آ}
\vspace*{3mm}


\farsiTOenglish{اسکالر}{Scalar}


\vspace*{3mm}
{\bf ب}
\vspace*{3mm}

\farsiTOenglish{بالابر}{Lift}


\vspace*{3mm}
{\bf پ}
%%\vspace*{3mm}

\farsiTOenglish{پایا}{Invariant}



\vspace*{3mm}
{\bf ت}
%%\vspace*{3mm}

\farsiTOenglish{ تناظر }{Correspondence}


\vspace*{3mm}
{\bf ث}
%%\vspace*{3mm}

\farsiTOenglish{ثابت‌ساز}{Stabilizer}

\vspace*{3mm}
{\bf ج}
%%\vspace*{3mm}

\farsiTOenglish{جایگشت}{Permutation}



\vspace*{3mm}
{\bf چ}
%%\vspace*{3mm}


\farsiTOenglish{چند جمله‌ای }{Polynomial}

\vspace*{3mm}
{\bf ح}
%%\vspace*{3mm}

\farsiTOenglish{حاصل‌ضرب دکارتی}{Cartesian product}


\vspace*{3mm}
{\bf خ}
%%\vspace*{3mm}

\farsiTOenglish{خودریختی}{Automorphism}

\vspace*{3mm}
{\bf د}
%%\vspace*{3mm}

\farsiTOenglish{درجه}{Degree}


\vspace*{3mm}
{\bf ر}
%%\vspace*{3mm}


\farsiTOenglish{ریزپردازنده}{microprocessor}


\vspace*{3mm}
{\bf ز}
%%\vspace*{3mm}


\farsiTOenglish{زیرمدول}{Submodule}


\vspace*{3mm}
{\bf س}
%%\vspace*{3mm}

\farsiTOenglish{سرشت}{Character}


\vspace*{3mm}
{\bf ص}
%%\vspace*{3mm}

\farsiTOenglish{صادقانه}{Faithful}

\vspace*{3mm}
{\bf ض}
%%\vspace*{3mm}

\farsiTOenglish{ضرب داخلی}{Inner product}

\vspace*{3mm}
{\bf ط}
%%\vspace*{3mm}


\farsiTOenglish{طوقه}{Loop}


\vspace*{3mm}
{\bf ظ}
%%\vspace*{3mm}


\farsiTOenglish{ظرفیت}{Valency}
 
\vspace*{3mm}
{\bf ع}
%%\vspace*{3mm}


\farsiTOenglish{عدم مجاورت}{Nonadjacency}



\vspace*{3mm}
{\bf ف}
%%\vspace*{3mm}

\farsiTOenglish{فضای برداری}{Vector space}



\vspace*{3mm}
{\bf ک}
%%\vspace*{3mm}

\farsiTOenglish{کاملاً تحویل‌پذیر}{Complete reducibility}


\vspace*{3mm}
{\bf گ}
%%\vspace*{3mm}


\farsiTOenglish{گراف}{Graph}



\vspace*{3mm}
{\bf م}
%%\vspace*{3mm}

\farsiTOenglish{ماتریس جایگشتی}{Permutation matrix }


\vspace*{3mm}
{\bf ن}
%%\vspace*{3mm}

\farsiTOenglish{ناهمبند}{Disconnected}


\vspace*{3mm}
{\bf و}
%%\vspace*{3mm}

\farsiTOenglish{وارون‌پذیر}{Invertible}


\vspace*{3mm}
{\bf ه}
%%\vspace*{3mm}

\farsiTOenglish{همبند}{Connected}



\vspace*{3mm}
{\bf ی}
%%\vspace*{3mm}

\farsiTOenglish{یال}{Edge}




\end{multicols*}%
%%%%%%%
\chapter*{ واژه‌نامه‌ی انگلیسی به فارسی}
\pagestyle{style9}
\lhead{\thepage}\rhead{واژه‌نامه‌ی انگلیسی به فارسی}
\addcontentsline{toc}{chapter}{واژه‌نامه‌ی انگلیسی به فارسی}

\LTRmulticolcolumns
\begin{multicols}{2}
{\hfill\bf  \lr{A}}
%%\vspace*{1.5mm}

\englishTOfarsi{Automorphism}{خودریختی}

\vspace*{3mm}
{\hfill\bf   \lr{B}}
%%\vspace*{1.5mm}

\englishTOfarsi{Bijection}{دوسویی}

\vspace*{3mm}
{\hfill\bf   \lr{C}}
%%\vspace*{1.5mm}

\englishTOfarsi{Cycle group}{گروه دوری}

\vspace*{3mm}
{\hfill\bf   \lr{D}}
%%\vspace*{1.5mm}

\englishTOfarsi{Degree}{درجه}

\vspace*{3mm}
{\hfill\bf   \lr{E}}
%%\vspace*{1.5mm}

\englishTOfarsi{Edge}{یال}

\vspace*{3mm}
{\hfill\bf   \lr{F}}
%%\vspace*{1.5mm}

\englishTOfarsi{Function}{تابع}

\vspace*{3mm}
{\hfill\bf   \lr{G}}
%%\vspace*{1.5mm}

\englishTOfarsi{Group}{گروه}

\vspace*{3mm}
{\hfill\bf   \lr{H}}
%%\vspace*{1.5mm}

\englishTOfarsi{Homomorphism}{همریختی}

\vspace*{3mm}
{\hfill\bf   \lr{I}}
%%\vspace*{1.5mm}

\englishTOfarsi{Invariant}{پایا}

\vspace*{3mm}
{\hfill\bf   \lr{L}}
%%\vspace*{1.5mm}

\englishTOfarsi{Lift}{بالابر}

\vspace*{3mm}
{\hfill\bf   \lr{M}}
%%\vspace*{1.5mm}

\englishTOfarsi{Module}{مدول}

\vspace*{3mm}
{\hfill\bf   \lr{N}}
%%\vspace*{1.5mm}

\englishTOfarsi{Natural map}{نگاشت طبیعی}

\vspace*{3mm}
{\hfill\bf   \lr{O}}
%%\vspace*{1.5mm}

\englishTOfarsi{One to One}{یک به یک}

\vspace*{3mm}
{\hfill\bf   \lr{P}}
%%\vspace*{1.5mm}

\englishTOfarsi{Permutation group}{گروه جایگشتی}

\vspace*{3mm}
{\hfill\bf   \lr{Q}}
%%\vspace*{1.5mm}

\englishTOfarsi{Quotient graph}{گراف خارج‌قسمتی}

 \vspace*{3mm}
{\hfill\bf   \lr{R}}
%%\vspace*{1.5mm}

\englishTOfarsi{Reducible}{تحویل پذیر}

\vspace*{3mm}
{\hfill\bf   \lr{S}}
%%\vspace*{1.5mm}

\englishTOfarsi{Sequence}{دنباله}

 \vspace*{3mm}
{\hfill\bf   \lr{T}}
%%\vspace*{1.5mm}

\englishTOfarsi{Trivial character}{سرشت بدیهی}

\vspace*{3mm}
{\hfill\bf   \lr{U}}
%%\vspace*{1.5mm}

\englishTOfarsi{Unique}{منحصربفرد}

\vspace*{3mm}
{\hfill\bf   \lr{V}}
%%\vspace*{1.5mm}

\englishTOfarsi{Vector space}{فضای برداری}
\end{multicols}
%--------------------------------------------------------------------------index(نمایه)
%اگر مایل به داشتن صفحه نمایه نیستید، خط زیر را غیر فعال کنید.
%\pagestyle{style7}
%\printindex
%\pagestyle{style7}
%%کلمات کلیدی انگلیسی
\latinkeywords{Write a 3 to 5 KeyWords is essential. Example: AUT, M.Sc., Ph. D,..}
%چکیده انگلیسی

\en-abstract{
This page is accurate translation from Persian abstract into English.
}
%%%%%%%%%%%%%%%%%%%%% کدهای زیر را تغییر ندهید.

\newpage
\thispagestyle{empty}
\begin{latin}
\section*{\LARGE\centering Abstract}

\een-abstract

\vspace*{.5cm}
{\large\textbf{Key Words:}}\par
\vspace*{.5cm}
\elatinkeywords
\end{latin}
%% در این فایل، عنوان پایان‌نامه، مشخصات خود و چکیده پایان‌نامه را به انگلیسی، وارد کنید.
%%%%%%%%%%%%%%%%%%%%%%%%%%%%%%%%%%%%
\baselineskip=.6cm
\begin{latin}

\latinfaculty{Department of ...}


\latintitle{Title of Thesis}


\firstlatinsupervisor{Dr. }

%\secondlatinsupervisor{Second Supervisor}

\firstlatinadvisor{Dr. }

%\secondlatinadvisor{Second Advisor}

\latinname{Name}

\latinsurname{Surname}

\latinthesisdate{Month \& Year}

\latinvtitle
\end{latin}

\end{document}
